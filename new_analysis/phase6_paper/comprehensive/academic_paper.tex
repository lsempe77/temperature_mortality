% Options for packages loaded elsewhere
% Options for packages loaded elsewhere
\PassOptionsToPackage{unicode}{hyperref}
\PassOptionsToPackage{hyphens}{url}
\PassOptionsToPackage{dvipsnames,svgnames,x11names}{xcolor}
%
\documentclass[
]{article}
\usepackage{xcolor}
\usepackage[margin=1in]{geometry}
\usepackage{amsmath,amssymb}
\setcounter{secnumdepth}{5}
\usepackage{iftex}
\ifPDFTeX
  \usepackage[T1]{fontenc}
  \usepackage[utf8]{inputenc}
  \usepackage{textcomp} % provide euro and other symbols
\else % if luatex or xetex
  \usepackage{unicode-math} % this also loads fontspec
  \defaultfontfeatures{Scale=MatchLowercase}
  \defaultfontfeatures[\rmfamily]{Ligatures=TeX,Scale=1}
\fi
\usepackage{lmodern}
\ifPDFTeX\else
  % xetex/luatex font selection
\fi
% Use upquote if available, for straight quotes in verbatim environments
\IfFileExists{upquote.sty}{\usepackage{upquote}}{}
\IfFileExists{microtype.sty}{% use microtype if available
  \usepackage[]{microtype}
  \UseMicrotypeSet[protrusion]{basicmath} % disable protrusion for tt fonts
}{}
\makeatletter
\@ifundefined{KOMAClassName}{% if non-KOMA class
  \IfFileExists{parskip.sty}{%
    \usepackage{parskip}
  }{% else
    \setlength{\parindent}{0pt}
    \setlength{\parskip}{6pt plus 2pt minus 1pt}}
}{% if KOMA class
  \KOMAoptions{parskip=half}}
\makeatother
% Make \paragraph and \subparagraph free-standing
\makeatletter
\ifx\paragraph\undefined\else
  \let\oldparagraph\paragraph
  \renewcommand{\paragraph}{
    \@ifstar
      \xxxParagraphStar
      \xxxParagraphNoStar
  }
  \newcommand{\xxxParagraphStar}[1]{\oldparagraph*{#1}\mbox{}}
  \newcommand{\xxxParagraphNoStar}[1]{\oldparagraph{#1}\mbox{}}
\fi
\ifx\subparagraph\undefined\else
  \let\oldsubparagraph\subparagraph
  \renewcommand{\subparagraph}{
    \@ifstar
      \xxxSubParagraphStar
      \xxxSubParagraphNoStar
  }
  \newcommand{\xxxSubParagraphStar}[1]{\oldsubparagraph*{#1}\mbox{}}
  \newcommand{\xxxSubParagraphNoStar}[1]{\oldsubparagraph{#1}\mbox{}}
\fi
\makeatother


\usepackage{longtable,booktabs,array}
\usepackage{calc} % for calculating minipage widths
% Correct order of tables after \paragraph or \subparagraph
\usepackage{etoolbox}
\makeatletter
\patchcmd\longtable{\par}{\if@noskipsec\mbox{}\fi\par}{}{}
\makeatother
% Allow footnotes in longtable head/foot
\IfFileExists{footnotehyper.sty}{\usepackage{footnotehyper}}{\usepackage{footnote}}
\makesavenoteenv{longtable}
\usepackage{graphicx}
\makeatletter
\newsavebox\pandoc@box
\newcommand*\pandocbounded[1]{% scales image to fit in text height/width
  \sbox\pandoc@box{#1}%
  \Gscale@div\@tempa{\textheight}{\dimexpr\ht\pandoc@box+\dp\pandoc@box\relax}%
  \Gscale@div\@tempb{\linewidth}{\wd\pandoc@box}%
  \ifdim\@tempb\p@<\@tempa\p@\let\@tempa\@tempb\fi% select the smaller of both
  \ifdim\@tempa\p@<\p@\scalebox{\@tempa}{\usebox\pandoc@box}%
  \else\usebox{\pandoc@box}%
  \fi%
}
% Set default figure placement to htbp
\def\fps@figure{htbp}
\makeatother


% definitions for citeproc citations
\NewDocumentCommand\citeproctext{}{}
\NewDocumentCommand\citeproc{mm}{%
  \begingroup\def\citeproctext{#2}\cite{#1}\endgroup}
\makeatletter
 % allow citations to break across lines
 \let\@cite@ofmt\@firstofone
 % avoid brackets around text for \cite:
 \def\@biblabel#1{}
 \def\@cite#1#2{{#1\if@tempswa , #2\fi}}
\makeatother
\newlength{\cslhangindent}
\setlength{\cslhangindent}{1.5em}
\newlength{\csllabelwidth}
\setlength{\csllabelwidth}{3em}
\newenvironment{CSLReferences}[2] % #1 hanging-indent, #2 entry-spacing
 {\begin{list}{}{%
  \setlength{\itemindent}{0pt}
  \setlength{\leftmargin}{0pt}
  \setlength{\parsep}{0pt}
  % turn on hanging indent if param 1 is 1
  \ifodd #1
   \setlength{\leftmargin}{\cslhangindent}
   \setlength{\itemindent}{-1\cslhangindent}
  \fi
  % set entry spacing
  \setlength{\itemsep}{#2\baselineskip}}}
 {\end{list}}
\usepackage{calc}
\newcommand{\CSLBlock}[1]{\hfill\break\parbox[t]{\linewidth}{\strut\ignorespaces#1\strut}}
\newcommand{\CSLLeftMargin}[1]{\parbox[t]{\csllabelwidth}{\strut#1\strut}}
\newcommand{\CSLRightInline}[1]{\parbox[t]{\linewidth - \csllabelwidth}{\strut#1\strut}}
\newcommand{\CSLIndent}[1]{\hspace{\cslhangindent}#1}



\setlength{\emergencystretch}{3em} % prevent overfull lines

\providecommand{\tightlist}{%
  \setlength{\itemsep}{0pt}\setlength{\parskip}{0pt}}



 


\usepackage{float}
\floatplacement{figure}{H}
\floatplacement{table}{H}
\usepackage{booktabs}
\usepackage{array}
\usepackage{multirow}
\renewcommand{\abstractname}{\large Summary}
\makeatletter
\@ifpackageloaded{caption}{}{\usepackage{caption}}
\AtBeginDocument{%
\ifdefined\contentsname
  \renewcommand*\contentsname{Table of contents}
\else
  \newcommand\contentsname{Table of contents}
\fi
\ifdefined\listfigurename
  \renewcommand*\listfigurename{List of Figures}
\else
  \newcommand\listfigurename{List of Figures}
\fi
\ifdefined\listtablename
  \renewcommand*\listtablename{List of Tables}
\else
  \newcommand\listtablename{List of Tables}
\fi
\ifdefined\figurename
  \renewcommand*\figurename{Figure}
\else
  \newcommand\figurename{Figure}
\fi
\ifdefined\tablename
  \renewcommand*\tablename{Table}
\else
  \newcommand\tablename{Table}
\fi
}
\@ifpackageloaded{float}{}{\usepackage{float}}
\floatstyle{ruled}
\@ifundefined{c@chapter}{\newfloat{codelisting}{h}{lop}}{\newfloat{codelisting}{h}{lop}[chapter]}
\floatname{codelisting}{Listing}
\newcommand*\listoflistings{\listof{codelisting}{List of Listings}}
\makeatother
\makeatletter
\makeatother
\makeatletter
\@ifpackageloaded{caption}{}{\usepackage{caption}}
\@ifpackageloaded{subcaption}{}{\usepackage{subcaption}}
\makeatother
\usepackage{bookmark}
\IfFileExists{xurl.sty}{\usepackage{xurl}}{} % add URL line breaks if available
\urlstyle{same}
\hypersetup{
  pdftitle={Temperature-Attributable Mortality Among Elderly Brazilians, 2010-2024: A National Time-Series Analysis},
  pdfauthor={Lucas Sempé},
  colorlinks=true,
  linkcolor={blue},
  filecolor={Maroon},
  citecolor={Blue},
  urlcolor={Blue},
  pdfcreator={LaTeX via pandoc}}


\title{Temperature-Attributable Mortality Among Elderly Brazilians,
2010-2024: A National Time-Series Analysis}
\author{Lucas Sempé}
\date{2025-12-01}
\begin{document}
\maketitle
\begin{abstract}
\textbf{Background:} Climate change is increasing temperature extremes
globally, yet comprehensive national evidence on temperature-mortality
relationships in large middle-income countries remains limited. Previous
studies in Brazil focused on single cities or short periods, and none
has systematically compared spatial aggregation scales or explicitly
quantified mortality displacement (harvesting) in elderly populations.

\textbf{Methods:} We analyzed 13.7 million elderly deaths (age ≥60
years) across two spatial scales---510 immediate geographic regions and
133 intermediate regions---covering all of Brazil from 2010 to 2024 (15
years). Using distributed lag non-linear models (DLNM) with a two-stage
meta-analytic design, we estimated temperature-mortality associations
with 21-day cumulative lag structure. We quantified heterogeneity (I²,
Cochran's Q), calculated attributable burden and years of life lost,
assessed mortality displacement via extended 35-day lag analysis, and
examined effect modification by age, sex, and cause of death.

\textbf{Findings:} Both extreme heat (P99) and cold (P1) significantly
increased mortality. At the intermediate level: heat RR = 1.088 (95\%
CI: 1.067-1.110); cold RR = 1.122 (1.098-1.146); I² = 51\%. At the
immediate level: heat RR = 1.063 (1.050-1.076); cold RR = 1.095
(1.085-1.106); I² = 29\%. Cold effects consistently exceeded heat
(cold:heat burden ratio 8:1), consistent with the Gasparrini et
al.~(2015) multi-country Lancet study that found 4:1 ratios in Brazilian
cities. Critically, harvesting analysis revealed heat deaths represent
true excess mortality (effects increase over extended lags), while cold
effects accumulate progressively over weeks. Annual burden:
\textasciitilde56,000 temperature-attributable elderly deaths, with cold
accounting for 89\%. The 80+ age group showed highest vulnerability
(cold RR 1.27); cardiovascular deaths showed 32\% cold-related excess;
females showed higher heat vulnerability (+8\%), males higher cold
vulnerability (+6\%).

\textbf{Interpretation:} This study provides the most comprehensive
national evidence on temperature-mortality in elderly Brazilians to
date. Cold effects substantially exceed heat effects, with an 8:1 burden
ratio reflecting elderly vulnerability---higher than the 4:1 ratio in
previous all-age studies. Public health strategies should address both
temperature extremes, with particular attention to cold protection for
elderly populations---a finding that challenges the predominant focus on
heat in tropical climate-health discourse.

\textbf{Funding:} {[}Funding sources{]}
\end{abstract}


\section{Research in Context}\label{research-in-context}

\subsection{Evidence before this
study}\label{evidence-before-this-study}

We searched PubMed for articles published between January 1, 2000 and
December 1, 2025, using terms ``Brazil'' AND ``temperature'' AND
``mortality'' AND (``cold'' OR ``heat''). We identified 40 relevant
studies. The landmark multi-country study by Gasparrini et al.~(2015,
Lancet) included Brazilian cities and found that cold-attributable
mortality (2.83\%) substantially exceeded heat-attributable mortality
(0.70\%) in Brazil---a pattern consistent across most countries
globally.\textsuperscript{1} The seminal São Paulo study by Gouveia et
al.~(2003) first established the U-shaped temperature-mortality
relationship in Brazil, with 5.5\% mortality increase per 1°C below
20°C.\textsuperscript{2} Subsequent studies confirmed cold dominance:
Son et al.~(2016) found cold effects (8.6\% excess) exceeded heat
effects (6.1\%) in São Paulo;\textsuperscript{3} Silveira et al.~(2019)
documented this pattern across 27 cities;\textsuperscript{4} Aschidamini
et al.~(2025) found national cold effects (RR 1.30 for elderly)
exceeding heat (RR 1.13).\textsuperscript{5} The recent Latin American
analysis by Kephart et al.~(2022, Nature Medicine) confirmed
temperature-mortality associations across the region.\textsuperscript{6}
However, existing studies had important limitations: (1) most focused on
single cities or states rather than national coverage; (2) none
systematically compared results across different spatial aggregation
scales; (3) few explicitly quantified mortality displacement
(harvesting); (4) none focused specifically on elderly populations with
comprehensive 15-year coverage.

\subsection{Added value of this study}\label{added-value-of-this-study}

This study provides the most comprehensive national analysis of
temperature-mortality in Brazil to date, with several novel
contributions. First, we analyze 13.7 million elderly deaths across 15
years (2010-2024), providing unprecedented statistical power. Second, we
systematically compare two spatial scales (133 intermediate vs 510
immediate regions), demonstrating that finer resolution reduces
heterogeneity (I² 29\% vs 51\%) while maintaining similar effect
estimates---important methodological guidance for future studies. Third,
we explicitly quantify harvesting, showing that heat deaths represent
true excess mortality (negative harvesting ratio) while cold effects
accumulate over weeks---a distinction critical for burden estimation and
policy. Fourth, we provide the first elderly-specific national burden
estimates: approximately 56,000 annual temperature-attributable deaths,
with an 8:1 cold-to-heat ratio higher than previous all-age estimates,
reflecting elderly vulnerability. Fifth, we demonstrate clear
vulnerability gradients by age (80+ most vulnerable), sex (females
heat-vulnerable, males cold-vulnerable), and cause (cardiovascular
deaths show 32\% cold excess).

\subsection{Implications of all the available
evidence}\label{implications-of-all-the-available-evidence}

The consistent finding across global and Brazilian studies that
cold-attributable mortality exceeds heat-attributable mortality has
profound implications for climate adaptation policy. In Brazil---a
country often perceived as facing primarily heat-related health risks
due to its tropical climate---cold protection for elderly populations
should receive equal or greater public health priority. Our harvesting
analysis adds nuance: while heat deaths are true excess (prevented
deaths = lives saved), cold effects accumulate over weeks through
cardiovascular and respiratory pathways. Climate change projections
suggest heat burden will increase while cold burden may decrease, but
current policy should address the dominant present-day burden. The 80+
age group and cardiovascular patients warrant targeted interventions
during both temperature extremes.

\section{Introduction}\label{introduction}

Climate change is increasing the frequency and intensity of temperature
extremes worldwide, with profound implications for human
health.\textsuperscript{1} Temperature-mortality relationships follow a
characteristic J- or U-shaped pattern, with increased mortality at both
cold and hot extremes relative to an optimal
temperature.\textsuperscript{7,8} Understanding these relationships is
critical for public health planning and climate adaptation.

A key finding from global temperature-mortality research is that
cold-attributable mortality substantially exceeds heat-attributable
mortality in most populations. The landmark multi-country study by
Gasparrini et al.~(2015), which included data from 13 countries and 74
million deaths, found that 7.29\% of mortality was attributable to cold
versus only 0.42\% to heat globally.\textsuperscript{1} In Brazil
specifically, that study found a 4:1 cold-to-heat burden ratio (2.83\%
vs 0.70\% attributable mortality). The seminal São Paulo study by
Gouveia et al.~(2003) first established this U-shaped relationship in
Brazil, with 5.5\% mortality increase per 1°C below 20°C and 2.6\% per
1°C above.\textsuperscript{2} Subsequent studies confirmed this pattern:
Son et al.~(2016) found cold effects (8.6\% excess) exceeded heat
effects (6.1\%) in São Paulo;\textsuperscript{3} Silveira et al.~(2019)
documented similar findings across 27 Brazilian
cities;\textsuperscript{4} and most recently, Aschidamini et al.~(2025)
reported national cold effects (RR 1.30 for elderly) exceeding heat (RR
1.13) in metropolitan areas.\textsuperscript{5}

Brazil presents a particularly important case for studying
temperature-mortality associations. As the world's fifth most populous
country and the largest in Latin America, Brazil has a rapidly aging
population---individuals aged 60 years and older now comprise over 15\%
of the population.\textsuperscript{9} Brazil's continental size spans
tropical, subtropical, and temperate climate zones, providing natural
variation in temperature exposures.\textsuperscript{6} The minimum
mortality temperature (MMT) in tropical Brazilian populations occurs at
approximately the 60th percentile of the temperature
distribution---considerably lower than the 80-90th percentile observed
in temperate regions---reflecting population adaptation to warmer
climates.\textsuperscript{1} Heat waves also pose significant risk, with
multi-country evidence showing mortality increases during prolonged heat
events,\textsuperscript{10} and Brazilian studies documenting heatwave
associations with hospitalization\textsuperscript{11} and
mortality.\textsuperscript{12}

Despite this substantial evidence base, existing Brazilian studies have
important limitations. Most focused on single
cities\textsuperscript{2,3} or subsets of cities\textsuperscript{4}
rather than providing truly national coverage. None has systematically
compared results across different spatial aggregation scales to assess
exposure misclassification. Few explicitly quantified mortality
displacement (harvesting) to distinguish true excess mortality from
short-term displacement.\textsuperscript{13} And none focused
specifically on elderly populations---the most vulnerable
group\textsuperscript{14}---with comprehensive long-term national
coverage.

This study addresses these gaps by quantifying the burden of
temperature-attributable mortality among elderly Brazilians (≥60 years)
using comprehensive national data spanning 15 years (2010-2024). We
employed distributed lag non-linear models (DLNM), the methodological
gold standard for temperature-mortality studies,\textsuperscript{15}
applied at two spatial scales: 510 immediate geographic regions and 133
intermediate regions. We specifically examined: (1) the overall
temperature-mortality relationship and comparison across spatial scales;
(2) heat- and cold-attributable deaths and years of life lost; (3)
mortality displacement (harvesting) to distinguish true excess from
displaced deaths; (4) effect modification by age, sex, and cause of
death; and (5) comparison of our findings with the existing literature.

\section{Methods}\label{methods}

\subsection{Study Design and
Population}\label{study-design-and-population}

We conducted a national time-series analysis of elderly mortality (age
≥60 years) across Brazil from January 1, 2010 to December 31, 2024 (15
complete years). The primary analysis used 510 immediate geographic
regions (regiões geográficas imediatas), with 133 intermediate regions
(regiões geográficas intermediárias) as a sensitivity analysis to assess
the impact of spatial aggregation. The study population included all
registered elderly deaths from natural causes (ICD-10 codes A00-R99).
COVID-19 pandemic effects (2020-2021) were controlled through inclusion
of SARI surveillance data as a covariate.

The choice of immediate regions as the primary spatial unit was
motivated by concerns about Berkson error---the attenuation of
exposure-response relationships when using coarser spatial aggregation
that averages over heterogeneous exposures within regions.

\subsection{Data Sources}\label{data-sources}

\textbf{Mortality data} were obtained from Brazil's Mortality
Information System (Sistema de Informação sobre Mortalidade, SIM),
maintained by the Ministry of Health. SIM achieves \textgreater95\%
registration coverage nationally and includes cause of death coded
according to ICD-10.\textsuperscript{16}

\textbf{Temperature data} were obtained from the ERA5 reanalysis product
(European Centre for Medium-Range Weather Forecasts), which provides
hourly estimates at 0.25° spatial resolution.\textsuperscript{17} Daily
mean temperature was calculated as the average of 24 hourly values.
Regional exposures were computed as population-weighted averages of grid
cells within each region, using 2022 Census population distributions.

\textbf{Covariate data} included air quality (PM\textsubscript{2.5},
PM\textsubscript{10}, O\textsubscript{3}) from the Copernicus Atmosphere
Monitoring Service (CAMS), severe acute respiratory infections (SARI)
from the SIVEP-Gripe surveillance system, and socioeconomic indicators
from the Brazilian Institute of Geography and Statistics (IBGE).

\subsection{Statistical Analysis}\label{statistical-analysis}

We applied a two-stage distributed lag non-linear model (DLNM)
design.\textsuperscript{18}

\textbf{First stage:} For each of the 510 immediate regions, we fitted
quasi-Poisson regression models:

\[\log[\mathbb{E}(Y_{r,t})] = \alpha_r + \text{cb}(T_{r,t}) + ns(t; 8\text{df/yr}) + \gamma \cdot \text{DOW}_t + \delta \cdot H_t + \log(P_r)\]

where \(Y_{r,t}\) is the daily death count in region \(r\) on day \(t\);
cb(\(T\)) is a cross-basis function for temperature with natural spline
for temperature (4 df, knots at 25th, 50th, 75th percentiles) and lag (4
df, maximum lag 21 days); \(ns(t)\) controls for long-term and seasonal
trends; DOW and \(H\) are indicators for day of week and holidays; and
\(P_r\) is the elderly population (offset).

\textbf{Second stage:} Region-specific estimates were pooled using
random-effects meta-analysis with restricted maximum likelihood (REML)
estimation to obtain national estimates.

\textbf{Attributable burden:} We calculated attributable fraction (AF)
and attributable number (AN) using established
methods.\textsuperscript{19} Following established methodology for
extreme temperature studies, heat-attributable mortality was defined as
deaths attributable to temperatures above the 97.5th percentile;
cold-attributable mortality as deaths below the 2.5th percentile. The
median temperature (50th percentile) served as reference. As sensitivity
analysis, we also computed burden using P1/P99 thresholds to capture the
most extreme temperatures. Years of life lost (YLL) were calculated by
multiplying AN by age-specific remaining life expectancy from IBGE life
tables.

\textbf{Harvesting analysis:} To assess mortality displacement, we
compared cumulative relative risks at 7-day versus 35-day lag horizons.
The harvesting ratio was calculated as:
\((ERR_{7d} - ERR_{35d})/ERR_{7d}\), where \(ERR = RR - 1\).

\textbf{Spatial aggregation sensitivity:} We repeated all analyses using
133 intermediate regions to assess whether coarser spatial aggregation
affects estimates---testing for potential Berkson error.

\textbf{Additional sensitivity analyses} included: varying lag structure
(7, 14, 21, 28 days); polynomial degrees (2, 3, 4 df); temporal
cross-validation (leave-one-year-out); COVID period exclusion; and
robust standard errors.

\textbf{Heterogeneity analyses} examined effect modification by age
group (60-69, 70-79, 80+), sex, and cause of death (cardiovascular,
respiratory, other).

All analyses were conducted using R 4.4.1 with packages dlnm, mixmeta
1.2.0, and data.table, following a two-stage design. Region-specific
DLNM models were pooled via multivariate random-effects meta-analysis
with REML estimation. Heterogeneity was assessed using Cochran's Q test
and the I² statistic. Statistical significance was defined as
p\textless0.05 (two-sided).

\section{Results}\label{results}

\subsection{Ethical Approval}\label{ethical-approval}

This study used publicly available, anonymized administrative data and
did not require ethical approval.

\subsection{Study Population}\label{study-population}

Over the 15-year study period (2010-2024), we analyzed 13,677,712
elderly deaths. At the intermediate level, 133 regions contributed
660,980 region-days of observation. At the immediate level, 510 regions
contributed 2,302,539 region-days. Mean daily temperature ranged from
11.5°C (1st percentile) to 30.1°C (99th percentile) across regions, with
a median of approximately 24°C.

\subsection{Temperature-Mortality
Association}\label{temperature-mortality-association}

The pooled exposure-response relationship showed the characteristic
J-shaped curve, with mortality increasing at both temperature extremes
relative to the minimum mortality temperature (MMT) (\textbf{Figure 1}).

\begin{figure}

\centering{

\includegraphics[width=1\linewidth,height=\textheight,keepaspectratio]{../../phase5_outputs/figures/fig1_pooled_exposure_response.png}

}

\caption{\label{fig-exposure-response}\textbf{Figure 1: Pooled
exposure-response relationship between temperature and mortality.} The
curve shows relative risk of mortality across the temperature
distribution, with the minimum mortality temperature (MMT) as reference.
Shading represents 95\% confidence intervals. Both cold (left) and heat
(right) extremes are associated with increased mortality risk.}

\end{figure}%

\textbf{Intermediate level (133 regions):} The MMT was 24.3°C. Extreme
heat (99th percentile, \textasciitilde30°C) was associated with an 8.8\%
increase in mortality risk (RR 1.088, 95\% CI: 1.067-1.110) compared to
the MMT. Extreme cold (1st percentile, \textasciitilde12°C) showed a
12.2\% increase in mortality risk (RR 1.122, 95\% CI: 1.098-1.146).
Heterogeneity was moderate (I² = 51.0\%, Cochran's Q = 4,309,
p\textless0.001).

\textbf{Immediate level (510 regions):} The MMT was lower at 22.6°C.
Heat effects were attenuated: RR 1.063 (95\% CI: 1.050-1.076, 6.3\%
excess mortality). Cold effects were similar: RR 1.095 (95\% CI:
1.085-1.106, 9.5\% excess). Heterogeneity was notably lower (I² =
29.3\%, Cochran's Q = 11,524, p\textless0.001), suggesting more
homogeneous exposure-response relationships at the finer spatial scale.

The intermediate level consistently showed \textasciitilde2.5\% higher
relative risks, likely reflecting reduced exposure misclassification at
larger spatial units where temperature averaging is less problematic.

\subsection{Attributable Burden}\label{attributable-burden}

\textbf{Intermediate level (133 regions):} Over the 15-year study
period, we estimated 89,941 annual deaths attributable to heat exposure
and 750,975 annual deaths attributable to cold exposure (based on
cumulative burden calculations). This translates to approximately 5,996
heat-attributable deaths and 50,065 cold-attributable deaths per year.
Cold accounts for 89.3\% of the total temperature-attributable burden
(\textbf{Figure 2}).

\begin{figure}

\centering{

\includegraphics[width=1\linewidth,height=\textheight,keepaspectratio]{../../phase5_outputs/figures/fig2_attributable_burden.png}

}

\caption{\label{fig-burden}\textbf{Figure 2: Temperature-attributable
mortality burden.} Annual deaths attributable to heat and cold exposure
at intermediate (133 regions) and immediate (510 regions) spatial
scales. Cold-attributable deaths substantially exceed heat-attributable
deaths at both scales.}

\end{figure}%

\textbf{Immediate level (510 regions):} Annual attributable deaths were
6,610 for heat and 34,692 for cold, with cold representing 84.0\% of the
burden. The lower total burden at immediate level reflects the
attenuated relative risks at finer spatial resolution.

\textbf{Years of Life Lost:} At the intermediate level, temperature
exposure accounts for approximately 718,223 years of life lost annually
(76,819 from heat + 641,404 from cold). The YLL rate is approximately
2,237 per 100,000 elderly population. Each temperature-attributable
death costs an average of 12.8 years of life expectancy (\textbf{Figure
3}).

\begin{figure}

\centering{

\includegraphics[width=1\linewidth,height=\textheight,keepaspectratio]{../../phase5_outputs/figures/fig3_yll_summary.png}

}

\caption{\label{fig-yll}\textbf{Figure 3: Years of life lost (YLL)
attributable to temperature.} Annual YLL from heat and cold exposure.
Cold accounts for the vast majority of temperature-related life years
lost.}

\end{figure}%

\textbf{Table 1} presents the main results for temperature-mortality
associations and attributable burden at both spatial scales.

\begin{longtable}[]{@{}lll@{}}
\caption{Temperature-attributable mortality burden among elderly
Brazilians, 2010-2024, by spatial aggregation
level}\label{tbl-main}\tabularnewline
\toprule\noalign{}
Metric & Intermediate (133 regions) & Immediate (510 regions) \\
\midrule\noalign{}
\endfirsthead
\toprule\noalign{}
Metric & Intermediate (133 regions) & Immediate (510 regions) \\
\midrule\noalign{}
\endhead
\bottomrule\noalign{}
\endlastfoot
\textbf{Temperature-mortality RR} & & \\
MMT (Optimal Temperature) & 24.3°C & 22.6°C \\
Heat (P99 vs MMT) & 1.088 {[}1.067-1.110{]} & 1.063 {[}1.050-1.076{]} \\
Cold (P1 vs MMT) & 1.122 {[}1.098-1.146{]} & 1.095 {[}1.085-1.106{]} \\
Heat (P95 vs MMT) & 1.011 {[}0.994-1.028{]} & 0.989 {[}0.978-0.999{]} \\
Cold (P5 vs MMT) & 1.052 {[}1.040-1.064{]} & 1.047 {[}1.041-1.053{]} \\
\textbf{Heterogeneity} & & \\
I² Statistic & 51.0\% & 29.3\% \\
Cochran's Q & 4,309 (df=2,112) & 11,524 (df=8,144) \\
Interpretation & Moderate & Low-moderate \\
\textbf{Annual Attributable Burden} & & \\
Heat deaths/year & 5,996 & 6,610 \\
Cold deaths/year & 50,065 & 34,692 \\
Total deaths/year & 56,061 & 41,302 \\
Cold share of burden & 89.3\% & 84.0\% \\
\textbf{Years of Life Lost} & & \\
YLL Heat/year & 76,819 & 84,686 \\
YLL Cold/year & 641,404 & 444,457 \\
YLL Rate (per 100k) & 2,237 & 1,648 \\
\end{longtable}

\subsection{Harvesting Analysis}\label{harvesting-analysis}

Analysis of mortality displacement (harvesting) revealed fundamentally
different temporal dynamics for heat versus cold effects (Table 2).
Extended lag analysis up to 35 days assessed whether temperature-related
deaths represent true excess mortality or displacement of already-frail
individuals.

\textbf{Heat Effects --- No Harvesting Detected:} The harvesting ratio
for heat was \emph{negative} (-0.37), meaning effects \emph{increased}
rather than diminished over longer time horizons. Heat deaths represent
true excess mortality, not displacement of individuals who would have
died soon anyway. This is consistent with acute physiological stress
mechanisms (hyperthermia, cardiovascular strain).

\textbf{Cold Effects --- Strong Delayed Mortality:} Cold showed
protective effects at short lags (7 days: RR 0.83) followed by
substantial delayed mortality accumulating over weeks. By 35 days, cold
RR reached 1.45 (45\% excess mortality). The high positive harvesting
ratio (+3.68) indicates cold effects are primarily delayed but
persistent---cold triggers cardiovascular stress, respiratory
infections, and inflammatory responses that manifest over weeks.

\begin{figure}

\centering{

\includegraphics[width=1\linewidth,height=\textheight,keepaspectratio]{../../phase5_outputs/figures/fig5_harvesting.png}

}

\caption{\label{fig-harvesting}\textbf{Figure 4: Harvesting analysis
comparing short-term and extended lag effects.} Comparison of
temperature-mortality associations at 7-day versus 35-day lag horizons
for heat and cold. Heat effects increase over time (negative
harvesting), while cold effects show delayed accumulation (positive
harvesting).}

\end{figure}%

\begin{longtable}[]{@{}
  >{\raggedright\arraybackslash}p{(\linewidth - 8\tabcolsep) * \real{0.1667}}
  >{\raggedright\arraybackslash}p{(\linewidth - 8\tabcolsep) * \real{0.1923}}
  >{\raggedright\arraybackslash}p{(\linewidth - 8\tabcolsep) * \real{0.2051}}
  >{\raggedright\arraybackslash}p{(\linewidth - 8\tabcolsep) * \real{0.2308}}
  >{\raggedright\arraybackslash}p{(\linewidth - 8\tabcolsep) * \real{0.2051}}@{}}
\caption{Mortality displacement (harvesting) analysis at intermediate
level (133 regions). Negative harvesting for heat indicates true excess
mortality; positive harvesting for cold indicates delayed
effects.}\label{tbl-harvest}\tabularnewline
\toprule\noalign{}
\begin{minipage}[b]{\linewidth}\raggedright
Temperature
\end{minipage} & \begin{minipage}[b]{\linewidth}\raggedright
ERR at 7 days
\end{minipage} & \begin{minipage}[b]{\linewidth}\raggedright
ERR at 35 days
\end{minipage} & \begin{minipage}[b]{\linewidth}\raggedright
Harvesting Ratio
\end{minipage} & \begin{minipage}[b]{\linewidth}\raggedright
Interpretation
\end{minipage} \\
\midrule\noalign{}
\endfirsthead
\toprule\noalign{}
\begin{minipage}[b]{\linewidth}\raggedright
Temperature
\end{minipage} & \begin{minipage}[b]{\linewidth}\raggedright
ERR at 7 days
\end{minipage} & \begin{minipage}[b]{\linewidth}\raggedright
ERR at 35 days
\end{minipage} & \begin{minipage}[b]{\linewidth}\raggedright
Harvesting Ratio
\end{minipage} & \begin{minipage}[b]{\linewidth}\raggedright
Interpretation
\end{minipage} \\
\midrule\noalign{}
\endhead
\bottomrule\noalign{}
\endlastfoot
Heat (P99) & +32.8\% & +45.0\% & -0.37 & Effects \emph{increase} over
time \\
Cold (P1) & -16.9\% (protective) & +45.3\% & +3.68 & Delayed but
persistent \\
\end{longtable}

\subsection{Effect Modification}\label{effect-modification}

Significant heterogeneity in temperature effects was observed across
subgroups (\textbf{Table 3}, \textbf{Figures 5-7}).

\textbf{Age:} Clear age gradient with vulnerability increasing with age
(\textbf{Figure 5}). At the intermediate level, the 80+ age group showed
the highest heat effects (RR 1.187, 95\% CI: 1.152-1.224) and cold
effects (RR 1.273, 95\% CI: 1.240-1.308). The 60-69 group showed lower
effects (heat RR 1.069; cold RR 1.171). This \textasciitilde12\% heat
difference and \textasciitilde10\% cold difference between oldest and
youngest elderly groups was statistically significant (p\textless0.005).

\begin{figure}

\centering{

\includegraphics[width=1\linewidth,height=\textheight,keepaspectratio]{../../phase5_outputs/figures/fig8_age_stratification.png}

}

\caption{\label{fig-age}\textbf{Figure 5: Effect modification by age
group.} Temperature-mortality associations stratified by age (60-69,
70-79, 80+ years). The oldest-old (80+) show the highest vulnerability
to both heat and cold extremes.}

\end{figure}%

\textbf{Sex:} Females showed higher heat vulnerability (RR 1.182 vs
1.100 for males, +8\% difference). Males showed higher cold
vulnerability (RR 1.251 vs 1.214 for females, +4-6\% difference). These
sex differences may reflect behavioral, physiological, or occupational
factors (\textbf{Figure 6}).

\begin{figure}

\centering{

\includegraphics[width=1\linewidth,height=\textheight,keepaspectratio]{../../phase5_outputs/figures/fig9_sex_stratification.png}

}

\caption{\label{fig-sex}\textbf{Figure 6: Effect modification by sex.}
Temperature-mortality associations stratified by sex. Females show
higher heat vulnerability while males show higher cold vulnerability.}

\end{figure}%

\textbf{Cause of Death:} Cardiovascular deaths showed the highest cold
vulnerability (32\% excess at intermediate level, RR 1.315). Respiratory
deaths showed elevated effects for both extremes (\textasciitilde13-25\%
excess). External causes showed minimal temperature association (RR
\textasciitilde1.01-1.06), confirming the specificity of temperature
effects to physiologically plausible pathways (\textbf{Figure 7}).

\begin{figure}

\centering{

\includegraphics[width=1\linewidth,height=\textheight,keepaspectratio]{../../phase5_outputs/figures/fig10_cause_stratification.png}

}

\caption{\label{fig-cause}\textbf{Figure 7: Effect modification by cause
of death.} Temperature-mortality associations stratified by cause
(cardiovascular, respiratory, external, other). Cardiovascular deaths
show the strongest cold associations; external causes show minimal
effects, confirming specificity.}

\end{figure}%

\begin{longtable}[]{@{}
  >{\raggedright\arraybackslash}p{(\linewidth - 8\tabcolsep) * \real{0.0901}}
  >{\raggedright\arraybackslash}p{(\linewidth - 8\tabcolsep) * \real{0.2432}}
  >{\raggedright\arraybackslash}p{(\linewidth - 8\tabcolsep) * \real{0.2342}}
  >{\raggedright\arraybackslash}p{(\linewidth - 8\tabcolsep) * \real{0.2162}}
  >{\raggedright\arraybackslash}p{(\linewidth - 8\tabcolsep) * \real{0.2162}}@{}}
\caption{Effect modification of temperature-mortality association by
age, sex, and cause of death at both spatial
levels.}\label{tbl-effect-mod}\tabularnewline
\toprule\noalign{}
\begin{minipage}[b]{\linewidth}\raggedright
Subgroup
\end{minipage} & \begin{minipage}[b]{\linewidth}\raggedright
Heat RR (P99) Intermediate
\end{minipage} & \begin{minipage}[b]{\linewidth}\raggedright
Cold RR (P1) Intermediate
\end{minipage} & \begin{minipage}[b]{\linewidth}\raggedright
Heat RR (P99) Immediate
\end{minipage} & \begin{minipage}[b]{\linewidth}\raggedright
Cold RR (P1) Immediate
\end{minipage} \\
\midrule\noalign{}
\endfirsthead
\toprule\noalign{}
\begin{minipage}[b]{\linewidth}\raggedright
Subgroup
\end{minipage} & \begin{minipage}[b]{\linewidth}\raggedright
Heat RR (P99) Intermediate
\end{minipage} & \begin{minipage}[b]{\linewidth}\raggedright
Cold RR (P1) Intermediate
\end{minipage} & \begin{minipage}[b]{\linewidth}\raggedright
Heat RR (P99) Immediate
\end{minipage} & \begin{minipage}[b]{\linewidth}\raggedright
Cold RR (P1) Immediate
\end{minipage} \\
\midrule\noalign{}
\endhead
\bottomrule\noalign{}
\endlastfoot
\textbf{Age group} & & & & \\
60-69 years & 1.069 {[}1.049-1.090{]} & 1.171 {[}1.141-1.202{]} & 1.029
{[}1.010-1.049{]} & 1.127 {[}1.099-1.155{]} \\
70-79 years & 1.087 {[}1.056-1.119{]} & 1.189 {[}1.155-1.225{]} & 1.064
{[}1.038-1.090{]} & 1.140 {[}1.111-1.170{]} \\
80+ years & 1.187 {[}1.152-1.224{]} & 1.273 {[}1.240-1.308{]} & 1.153
{[}1.130-1.176{]} & 1.251 {[}1.226-1.276{]} \\
\textbf{Sex} & & & & \\
Male & 1.100 {[}1.078-1.123{]} & 1.251 {[}1.220-1.283{]} & 1.081
{[}1.063-1.100{]} & 1.184 {[}1.160-1.209{]} \\
Female & 1.182 {[}1.147-1.218{]} & 1.214 {[}1.186-1.242{]} & 1.156
{[}1.132-1.179{]} & 1.189 {[}1.167-1.212{]} \\
\textbf{Cause} & & & & \\
Cardiovascular & 1.102 & 1.315 & 1.059 & 1.195 \\
Respiratory & 1.134 & 1.247 & 1.103 & 1.207 \\
External & 1.014 & 1.062 & 1.000 & 1.059 \\
Other & 1.139 & 1.202 & 1.115 & 1.158 \\
\end{longtable}

\subsection{Sensitivity Analyses}\label{sensitivity-analyses}

Results were robust to analytical choices (\textbf{Table 4}).

\textbf{Lag Structure:} Varying maximum lag from 7 to 28 days showed
that heat effects peak at shorter lags (7-14 days) and decline at longer
horizons, while cold effects accumulate progressively. At the
intermediate level: heat P99 RR ranged from 1.181 (7 days) to 1.147 (28
days); cold P1 RR ranged from 1.080 (7 days) to 1.279 (28 days). The
21-day baseline captures most cumulative effects for both extremes.

\textbf{Heatwave Effects:} Multi-day heat events (≥3 consecutive days
\textgreater P95) showed an additive mortality effect of RR 1.011 (95\%
CI: 1.005-1.017), indicating \textasciitilde1.1\% additional risk beyond
single-day temperature effects (\textbf{Figure 8}). This supports public
health messaging about cumulative heat exposure during prolonged events.

\begin{figure}

\centering{

\includegraphics[width=1\linewidth,height=\textheight,keepaspectratio]{../../phase5_outputs/figures/fig6_heatwave_effects.png}

}

\caption{\label{fig-heatwave}\textbf{Figure 8: Heatwave effects on
mortality.} Additional mortality risk associated with multi-day heat
events beyond single-day temperature effects. Duration and intensity of
heatwaves both contribute to excess mortality.}

\end{figure}%

\textbf{Confounding Assessment:} Adjustment for air pollution (PM2.5,
O3) produced minimal change in estimates (±0.2\%), suggesting pollution
may be a mediator rather than confounder. Results were robust to
influenza adjustment (±0\% change) and apparent temperature
(humidity-adjusted) showed similar or slightly higher effects.

\textbf{Spatial Aggregation:} The comparison between immediate (510) and
intermediate (133) levels showed consistent effect estimates but notably
lower heterogeneity at finer resolution (I² 29\% vs 51\%), supporting
the methodological value of finer spatial units.

\begin{longtable}[]{@{}
  >{\raggedright\arraybackslash}p{(\linewidth - 6\tabcolsep) * \real{0.3043}}
  >{\raggedright\arraybackslash}p{(\linewidth - 6\tabcolsep) * \real{0.2319}}
  >{\raggedright\arraybackslash}p{(\linewidth - 6\tabcolsep) * \real{0.2319}}
  >{\raggedright\arraybackslash}p{(\linewidth - 6\tabcolsep) * \real{0.2319}}@{}}
\caption{Sensitivity analyses for temperature-mortality
associations.}\label{tbl-sensitivity}\tabularnewline
\toprule\noalign{}
\begin{minipage}[b]{\linewidth}\raggedright
Sensitivity Analysis
\end{minipage} & \begin{minipage}[b]{\linewidth}\raggedright
Heat RR Change
\end{minipage} & \begin{minipage}[b]{\linewidth}\raggedright
Cold RR Change
\end{minipage} & \begin{minipage}[b]{\linewidth}\raggedright
Interpretation
\end{minipage} \\
\midrule\noalign{}
\endfirsthead
\toprule\noalign{}
\begin{minipage}[b]{\linewidth}\raggedright
Sensitivity Analysis
\end{minipage} & \begin{minipage}[b]{\linewidth}\raggedright
Heat RR Change
\end{minipage} & \begin{minipage}[b]{\linewidth}\raggedright
Cold RR Change
\end{minipage} & \begin{minipage}[b]{\linewidth}\raggedright
Interpretation
\end{minipage} \\
\midrule\noalign{}
\endhead
\bottomrule\noalign{}
\endlastfoot
Lag 7 days (vs 21) & +5.0\% & -12.5\% & Heat peaks early, cold
accumulates \\
Lag 28 days (vs 21) & -0.5\% & +4.5\% & Cold continues to accumulate \\
With PM2.5/O3 & -0.2\% & -0.2\% & Pollution is mediator, not
confounder \\
With SARI/Influenza & ±0\% & ±0\% & Cold effect not confounded by flu \\
Apparent Temperature & +0.5\% & +0.5\% & Humidity has minor additional
effect \\
Immediate vs Intermediate & -2.5\% & -2.7\% & Consistent across spatial
scales \\
\end{longtable}

\section{Discussion}\label{discussion}

This comprehensive national analysis of 13.7 million elderly deaths in
Brazil reveals substantial mortality burden attributable to non-optimal
temperatures, with several key findings relevant to public health and
climate adaptation.

\subsection{Cold dominates the temperature-mortality
burden}\label{cold-dominates-the-temperature-mortality-burden}

Our finding that cold-related mortality vastly exceeds heat-related
mortality (8:1 ratio at intermediate level) is consistent with, and
extends, the existing evidence base. In the landmark Multi-Country
Multi-City study, Gasparrini et al.~found that cold caused 7.29\% of
deaths globally versus 0.42\% for heat---a 17:1
ratio.\textsuperscript{1} For Brazil specifically, they reported cold
mortality (2.83\%) exceeding heat (0.70\%) by a 4:1 ratio. This pattern
has been replicated across Brazilian studies: Gouveia et al.~(2003)
first documented it in São Paulo;\textsuperscript{2} Son et al.~(2016)
found 8.6\% cold vs 6.1\% heat effects;\textsuperscript{3} Silveira et
al.~(2019) confirmed it across 27 cities;\textsuperscript{4} and
Aschidamini et al.~(2025) recently reported national cold RR 1.30 vs
heat RR 1.13 for elderly.\textsuperscript{5} Tobías et al.~(2024)
documented the economic burden implications, showing cold causes \$2.1
billion annual losses in Central/South America versus \$290.7 million
for heat.\textsuperscript{20}

Our higher cold-to-heat ratio (8:1 compared to 4:1 in earlier studies)
likely reflects three factors: (1) our elderly-only population, which
shows heightened vulnerability to cold as documented in systematic
reviews;\textsuperscript{1,14} (2) our 21-day lag structure capturing
the full temporal evolution of cold effects, which extend beyond shorter
observation windows; and (3) our national scope including Brazil's
southern states where cold exposure is substantial.

\subsection{Harvesting distinguishes heat from cold
effects}\label{harvesting-distinguishes-heat-from-cold-effects}

A critical finding is the markedly different temporal dynamics of heat
versus cold mortality, with important implications for burden
estimation. Heat-related deaths showed 74\% harvesting---most represent
short-term displacement of deaths among already-frail individuals. The
negative harvesting ratio (-0.37) indicates that heat effects actually
\emph{increase} over longer time horizons, meaning heat interventions
save lives that would otherwise be permanently lost. In contrast, cold
effects persisted and amplified (ERR increased from 5.3\% at 7 days to
27.0\% at 35 days), with positive harvesting ratio (+3.68) indicating
true excess mortality. These findings support previous harvesting
analyses\textsuperscript{13,21} and underscore that raw heat death
counts substantially overestimate long-term impact while cold effects
may be underestimated with shorter lag windows.

\subsection{Spatial resolution affects heterogeneity, not effect
magnitude}\label{spatial-resolution-affects-heterogeneity-not-effect-magnitude}

Using finer spatial resolution (510 immediate regions vs 133
intermediate regions) substantially reduced heterogeneity (I² 29\% vs
51\%) while yielding similar point estimates. This supports theoretical
predictions that coarser spatial aggregation introduces additional
variance from averaging heterogeneous exposures. The intermediate level
is preferred for national policy estimates (higher precision, better
convergence in extended lag models), while immediate level is valuable
for urban planning (local estimates, lower heterogeneity).

\subsection{Clear vulnerability gradients support targeted
interventions}\label{clear-vulnerability-gradients-support-targeted-interventions}

The oldest-old (80+) show the highest vulnerability (heat RR 1.19, cold
RR 1.27 at intermediate level), with effects \textasciitilde12\% higher
than the 60-69 group, consistent with global evidence of age-related
thermoregulatory decline.\textsuperscript{1,14} Sex differences are
pronounced: females show 8\% higher heat vulnerability while males show
4-6\% higher cold vulnerability, consistent with Son et al.~(2016) who
found females more vulnerable to heat in São Paulo.\textsuperscript{3}
Cardiovascular deaths show the strongest cold associations (32\%
excess), supporting known mechanisms of cold-induced vasoconstriction,
increased blood pressure, and thrombosis, as documented in the Global
Burden of Disease cause-specific analysis.\textsuperscript{8}

\subsection{Respiratory vulnerability is
pronounced}\label{respiratory-vulnerability-is-pronounced}

Respiratory deaths showed the strongest associations with cold (RR 1.51
at P2.5 compared to 1.08 for cardiovascular), with substantial
attributable burden despite comprising only 14\% of total mortality.
This is consistent with Jacobson et al.~(2021), who found 27\%
heat-related and 16\% cold-related excess respiratory mortality risk in
27 Brazilian cities, and with Zhao et al.~(2019), who documented
heat-COPD hospitalization associations
nationally.\textsuperscript{22,23} Importantly, adjustment for influenza
season did not attenuate cold effects, indicating that the
cold-respiratory relationship is not simply confounding by winter flu,
consistent with the finding of Gasparrini et al.~that most cold deaths
are attributable to moderately cold rather than extreme cold
temperatures.\textsuperscript{1}

\subsection{Heatwave duration matters}\label{heatwave-duration-matters}

Multi-day heat events carry an additional 1.1\% mortality risk beyond
single-day temperature effects, consistent with the multi-country
heatwave study by Guo et al.~(2017)\textsuperscript{10} and
Brazilian-specific evidence from Moraes et al.~(2021) and Zhao et
al.~(2019).\textsuperscript{11,12} This supports public health messaging
about cumulative exposure during prolonged heatwaves and the need for
sustained intervention during heat emergencies.

\subsection{Strengths and limitations}\label{strengths-and-limitations}

\textbf{Strengths} of this study include the comprehensive national
scope covering 133 intermediate or 510 immediate regions over 15 years
(2010-2024), the robust two-stage DLNM design with random-effects
meta-analysis following established methodology,\textsuperscript{1,13}
explicit quantification of heterogeneity via Cochran's Q and I²
statistics, comparison across two spatial aggregation levels, detailed
harvesting analysis distinguishing true excess from displaced mortality,
and extensive sensitivity and stratification analyses confirming
robustness.

\textbf{Limitations} include potential exposure misclassification from
using ERA5 reanalysis temperature data at regional centroids rather than
individual-level exposures, though validation against ground stations
showed excellent agreement (r=0.95). The two-stage design, while
standard for DLNM studies, may not fully capture spatial dependencies
between regions. We focused on elderly mortality and cannot generalize
findings to younger populations. Some sensitivity analyses at the
immediate level showed convergence issues due to sparse data per region,
making the intermediate level more reliable for extended lag and
harvesting analyses.

\section{Conclusions}\label{conclusions}

Non-optimal temperatures cause substantial but largely preventable
mortality burden among elderly Brazilians. Cold effects vastly exceed
heat effects, accounting for 84-89\% of temperature-attributable deaths
(\textasciitilde50,000 annually vs \textasciitilde6,000 for heat at
intermediate level)---a pattern consistent with global
evidence\textsuperscript{1,24} and prior Brazilian
studies\textsuperscript{2,4,5} but with higher cold-to-heat ratios (8:1)
reflecting the heightened vulnerability of elderly
populations.\textsuperscript{14} The harvesting analysis reveals that
heat deaths represent true excess mortality (negative harvesting ratio),
while cold effects show delayed but persistent mortality.

Key public health implications include:

\begin{enumerate}
\def\labelenumi{\arabic{enumi}.}
\tightlist
\item
  \textbf{Reframe climate-health narrative:} Cold is the dominant
  temperature-mortality burden in Brazil despite its tropical climate,
  consistent with global evidence\textsuperscript{1} but contrary to
  common assumptions about tropical countries
\item
  \textbf{Target the oldest-old (80+):} Highest vulnerability group with
  15-27\% excess mortality at temperature extremes\\
\item
  \textbf{Develop cold warning systems:} Currently underemphasized
  despite 8× higher burden than heat
\item
  \textbf{Prioritize cardiovascular patients:} Strongest cold effects
  (32\% excess mortality) reflecting thermoregulatory
  stress\textsuperscript{8}
\item
  \textbf{Address sex differences:} Heat interventions for females, cold
  protection for males\textsuperscript{3}
\end{enumerate}

Our comparison of spatial scales demonstrates that intermediate regions
(133) provide more stable national estimates with moderate heterogeneity
(I² 51\%), while immediate regions (510) offer local estimates with
lower heterogeneity (I² 29\%). Climate change will likely shift this
balance toward greater heat burden,\textsuperscript{25} making current
cold mortality patterns a critical baseline against which future changes
can be measured. This study provides the first comprehensive,
elderly-specific, harvesting-adjusted estimates of the
temperature-mortality relationship in Brazil, establishing an evidence
base for targeted climate adaptation in aging populations.

\section{References}\label{references}

\phantomsection\label{refs}
\begin{CSLReferences}{0}{1}
\bibitem[\citeproctext]{ref-gasparrini2015mortality}
\CSLLeftMargin{1 }%
\CSLRightInline{Gasparrini A, Guo Y, Hashizume M, \emph{et al.}
Mortality risk attributable to high and low ambient temperature: A
multicountry observational study. \emph{The Lancet} 2015; \textbf{386}:
369--75.}

\bibitem[\citeproctext]{ref-gouveia2003socioeconomic}
\CSLLeftMargin{2 }%
\CSLRightInline{Gouveia N, Hajat S, Armstrong B. Socioeconomic
differentials in the temperature-mortality relationship in são paulo,
brazil. \emph{International Journal of Epidemiology} 2003; \textbf{32}:
390--7.}

\bibitem[\citeproctext]{ref-son2016temperature}
\CSLLeftMargin{3 }%
\CSLRightInline{Son J-Y, Gouveia N, Bravo MA, Freitas CU de, Bell ML.
The impact of temperature on mortality in a subtropical city: Effects of
cold, heat, and heat waves in são paulo, brazil. \emph{International
Journal of Biometeorology} 2016; \textbf{60}: 113--21.}

\bibitem[\citeproctext]{ref-silveira2019cardiovascular}
\CSLLeftMargin{4 }%
\CSLRightInline{Silveira IH, Oliveira BF de, Cortes TR, Junger WL. The
effect of ambient temperature on cardiovascular mortality in 27
brazilian cities. \emph{Science of the Total Environment} 2019;
\textbf{691}: 996--1004.}

\bibitem[\citeproctext]{ref-aschidamini2025modifiers}
\CSLLeftMargin{5 }%
\CSLRightInline{Aschidamini C, Leon ACMP de. Effect modifiers of the
temperature-mortality association for general and older adults
population of brazil's metropolitan areas. \emph{Cadernos de Saúde
Pública} 2025; \textbf{41}: e00042524.}

\bibitem[\citeproctext]{ref-kephart2022latam}
\CSLLeftMargin{6 }%
\CSLRightInline{Kephart JL, Sánchez BN, Moore JN, \emph{et al.}
City-level impact of extreme temperatures and mortality in latin
america. \emph{Nature Medicine} 2022; \textbf{28}: 1700--5.}

\bibitem[\citeproctext]{ref-zhao2021global}
\CSLLeftMargin{7 }%
\CSLRightInline{Zhao Q, Guo Y, Ye T, \emph{et al.} Global, regional, and
national burden of mortality associated with non-optimal ambient
temperatures from 2000 to 2019: A three-stage modelling study. \emph{The
Lancet Planetary Health} 2021; \textbf{5}: e415--25.}

\bibitem[\citeproctext]{ref-burkart2021causespecific}
\CSLLeftMargin{8 }%
\CSLRightInline{Burkart K, Brauer M, Aravkin AY, \emph{et al.}
Estimating the cause-specific relative risks of non-optimal temperature
on daily mortality: A two-part modelling approach applied to the global
burden of disease study. \emph{The Lancet} 2021; \textbf{398}: 685--97.}

\bibitem[\citeproctext]{ref-ibge2022}
\CSLLeftMargin{9 }%
\CSLRightInline{Instituto Brasileiro de Geografia e Estatística. Censo
demográfico 2022: Resultados do universo. 2023.}

\bibitem[\citeproctext]{ref-guo2017heatwave}
\CSLLeftMargin{10 }%
\CSLRightInline{Guo Y, Gasparrini A, Armstrong B, \emph{et al.} Heat
wave and mortality: A multicountry, multicommunity study.
\emph{Environmental Health Perspectives} 2017; \textbf{125}: 087006.}

\bibitem[\citeproctext]{ref-zhao2019heatwave}
\CSLLeftMargin{11 }%
\CSLRightInline{Zhao Q, Li S, Coelho M de SZS, \emph{et al.} The
association between heatwaves and risk of hospitalization in brazil: A
nationwide time series study between 2000 and 2015. \emph{PLoS Medicine}
2019; \textbf{16}: e1002753.}

\bibitem[\citeproctext]{ref-moraes2021heatwaves}
\CSLLeftMargin{12 }%
\CSLRightInline{Moraes SL de, Almendra R, Barrozo LV. Impact of heat
waves and cold spells on cause-specific mortality in the city of são
paulo, brazil. \emph{International Journal of Hygiene and Environmental
Health} 2021; \textbf{239}: 113861.}

\bibitem[\citeproctext]{ref-armstrong2017role}
\CSLLeftMargin{13 }%
\CSLRightInline{Armstrong B, Sera F, Vicedo-Cabrera AM, \emph{et al.}
The role of humidity in associations of high temperature with mortality:
A multicentre, multicountry study. \emph{Environmental Health
Perspectives} 2019; \textbf{127}: 097007.}

\bibitem[\citeproctext]{ref-bunker2016effects}
\CSLLeftMargin{14 }%
\CSLRightInline{Bunker A, Wildenhain J, Vandenbergh A, \emph{et al.}
Effects of air temperature on climate-sensitive mortality and morbidity
outcomes in the elderly; a systematic review and meta-analysis of
epidemiological evidence. \emph{EBioMedicine} 2016; \textbf{6}:
258--68.}

\bibitem[\citeproctext]{ref-gasparrini2014modeling}
\CSLLeftMargin{15 }%
\CSLRightInline{Gasparrini A. Modeling exposure--lag--response
associations with distributed lag non-linear models. \emph{Statistics in
Medicine} 2014; \textbf{33}: 881--99.}

\bibitem[\citeproctext]{ref-franca2008ill}
\CSLLeftMargin{16 }%
\CSLRightInline{França E, Abreu DX de, Rao C, Lopez AD. Ill-defined
causes of death in brazil: A redistribution method based on the
investigation of such causes. \emph{Bulletin of the World Health
Organization} 2008; \textbf{86}: 39--45.}

\bibitem[\citeproctext]{ref-hersbach2020era5}
\CSLLeftMargin{17 }%
\CSLRightInline{Hersbach H, Bell B, Berrisford P, \emph{et al.} The ERA5
global reanalysis. \emph{Quarterly Journal of the Royal Meteorological
Society} 2020; \textbf{146}: 1999--2049.}

\bibitem[\citeproctext]{ref-gasparrini2010distributed}
\CSLLeftMargin{18 }%
\CSLRightInline{Gasparrini A. Distributed lag linear and non-linear
models in r: The package dlnm. \emph{Journal of Statistical Software}
2011; \textbf{43}: 1--20.}

\bibitem[\citeproctext]{ref-gasparrini2014attributable}
\CSLLeftMargin{19 }%
\CSLRightInline{Gasparrini A, Leone M. Attributable risk from
distributed lag models. \emph{BMC Medical Research Methodology} 2014;
\textbf{14}: 1--8.}

\bibitem[\citeproctext]{ref-tobias2024economic}
\CSLLeftMargin{20 }%
\CSLRightInline{Tobías A, Íñiguez C, Díaz MH, \emph{et al.} Mortality
burden and economic loss attributable to cold and heat in central and
south america. \emph{Environmental Epidemiology} 2024; \textbf{8}:
e335.}

\bibitem[\citeproctext]{ref-schwartz2005harvesting}
\CSLLeftMargin{21 }%
\CSLRightInline{Schwartz J. Harvesting and long term exposure effects in
the relation between air pollution and mortality. \emph{American Journal
of Epidemiology} 2005; \textbf{161}: 585--94.}

\bibitem[\citeproctext]{ref-jacobson2021respiratory}
\CSLLeftMargin{22 }%
\CSLRightInline{Jacobson L da SV, Oliveira BF de, Schneider R,
Gasparrini A, Hacon S de S. Mortality risk from respiratory diseases due
to non-optimal temperature among brazilian elderlies.
\emph{International Journal of Environmental Research and Public Health}
2021; \textbf{18}: 5550.}

\bibitem[\citeproctext]{ref-zhao2019copd}
\CSLLeftMargin{23 }%
\CSLRightInline{Zhao Q, Li S, Coelho M de SZS, \emph{et al.} Ambient
heat and hospitalisation for COPD in brazil: A nationwide case-crossover
study. \emph{Thorax} 2019; \textbf{74}: 1031--6.}

\bibitem[\citeproctext]{ref-masselot2023excess}
\CSLLeftMargin{24 }%
\CSLRightInline{Masselot P, Mistry M, Vanoli J, \emph{et al.} Excess
mortality attributed to heat and cold: A health impact assessment study
in 854 cities in europe. \emph{The Lancet Planetary Health} 2023;
\textbf{7}: e271--81.}

\bibitem[\citeproctext]{ref-gasparrini2017projections}
\CSLLeftMargin{25 }%
\CSLRightInline{Gasparrini A, Guo Y, Sera F, \emph{et al.} Projections
of temperature-related excess mortality under climate change scenarios.
\emph{The Lancet Planetary Health} 2017; \textbf{1}: e360--7.}

\end{CSLReferences}

\newpage

\section{Appendix}\label{appendix}

\subsection{A1: Spatial Aggregation
Comparison}\label{a1-spatial-aggregation-comparison}

A key methodological contribution of this study is the comparison of
results at two spatial scales: 133 intermediate regions and 510
immediate regions.

\begin{longtable}[]{@{}
  >{\raggedright\arraybackslash}p{(\linewidth - 6\tabcolsep) * \real{0.1333}}
  >{\raggedright\arraybackslash}p{(\linewidth - 6\tabcolsep) * \real{0.3167}}
  >{\raggedright\arraybackslash}p{(\linewidth - 6\tabcolsep) * \real{0.2833}}
  >{\raggedright\arraybackslash}p{(\linewidth - 6\tabcolsep) * \real{0.2667}}@{}}
\toprule\noalign{}
\begin{minipage}[b]{\linewidth}\raggedright
Metric
\end{minipage} & \begin{minipage}[b]{\linewidth}\raggedright
Intermediate (133)
\end{minipage} & \begin{minipage}[b]{\linewidth}\raggedright
Immediate (510)
\end{minipage} & \begin{minipage}[b]{\linewidth}\raggedright
Interpretation
\end{minipage} \\
\midrule\noalign{}
\endhead
\bottomrule\noalign{}
\endlastfoot
Region-days & 660,980 & 2,302,539 & 3.5× more observations at finer
scale \\
MMT (Optimal Temp) & 24.3°C & 22.6°C & Lower MMT at finer scale \\
Heat RR (P99) & 1.088 {[}1.067-1.110{]} & 1.063 {[}1.050-1.076{]} &
Similar point estimates \\
Cold RR (P1) & 1.122 {[}1.098-1.146{]} & 1.095 {[}1.085-1.106{]} &
Similar point estimates \\
Heat RR (P95) & 1.011 {[}0.994-1.028{]} & 0.989 {[}0.978-0.999{]} &
Moderate heat effects similar \\
Cold RR (P5) & 1.052 {[}1.040-1.064{]} & 1.047 {[}1.041-1.053{]} &
Moderate cold effects similar \\
I² Statistic & 51.0\% & 29.3\% & Lower heterogeneity at finer scale \\
Cochran's Q & 4,309 (df=2,112) & 11,524 (df=8,144) & Both significant
(p\textless0.001) \\
Heat deaths/year & 5,996 & 6,610 & Similar burden \\
Cold deaths/year & 50,065 & 34,692 & Higher at coarser scale \\
YLL Heat/year & 76,819 & 84,686 & Similar \\
YLL Cold/year & 641,404 & 444,457 & Higher at coarser scale \\
\end{longtable}

The moderate heterogeneity at intermediate level (I² 51\%) and
low-moderate at immediate level (I² 29\%) both support the
random-effects meta-analysis approach. The intermediate level is
recommended for national policy due to better convergence in extended
lag analyses, while immediate level provides more local estimates with
reduced heterogeneity.

\subsection{A2: Extended Methods}\label{a2-extended-methods}

\subsubsection{A2.1 DLNM Specification
Details}\label{a2.1-dlnm-specification-details}

The cross-basis function combines exposure-response and lag-response
components:

\[\text{cb}(T_{r,t}) = \sum_{l=0}^{L} f(T_{r,t-l}) \times w(l)\]

where: - \(f(\cdot)\): Natural cubic spline with 4 df (knots at P25,
P50, P75) - \(w(\cdot)\): Natural cubic spline with 4 df (knots at equal
intervals in log-scale) - \(L = 21\) days maximum lag

\subsubsection{A2.2 Attributable Burden
Calculation}\label{a2.2-attributable-burden-calculation}

For heat (temperatures above P97.5):

\[AF_{heat} = \sum_{t: T_t > P97.5} \frac{RR(T_t) - 1}{RR(T_t)} / n_{days}\]

\[AN_{heat} = AF_{heat} \times \sum_{t: T_t > P97.5} Y_t\]

For cold (temperatures below P2.5):

\[AF_{cold} = \sum_{t: T_t < P2.5} \frac{RR(T_t) - 1}{RR(T_t)} / n_{days}\]

\[AN_{cold} = AF_{cold} \times \sum_{t: T_t < P2.5} Y_t\]

\subsubsection{A2.3 Harvesting
Adjustment}\label{a2.3-harvesting-adjustment}

Heat burden adjusted for mortality displacement:

\[AN_{heat,adjusted} = AN_{heat,raw} \times (1 - Harvesting Ratio)\]

where Harvesting Ratio = \((ERR_{7d} - ERR_{35d})/ERR_{7d}\) = 0.74
(74\%)

\subsubsection{A2.4 Years of Life Lost}\label{a2.4-years-of-life-lost}

\[YLL = AN \times \bar{LE}_{weighted}\]

where \(\bar{LE}_{weighted} = \sum_a w_a \times LE_a = 11.44\) years

Age weights (\(w_a\)) derived from observed elderly death distribution
in SIM.

\subsection{A3: Supplementary Tables}\label{a3-supplementary-tables}

\subsubsection{Table S1: Temperature Distribution by
Region}\label{table-s1-temperature-distribution-by-region}

\begin{longtable}[]{@{}lll@{}}
\toprule\noalign{}
Statistic & Value & Use \\
\midrule\noalign{}
\endhead
\bottomrule\noalign{}
\endlastfoot
P1 (Extreme cold) & 11.5°C & Extreme cold threshold \\
P5 (Moderate cold) & \textasciitilde14°C & Moderate cold reference \\
P25 & \textasciitilde21°C & Spline knot \\
P50 (Median) & \textasciitilde24°C & Reference point \\
P75 & \textasciitilde27°C & Spline knot \\
P95 (Moderate heat) & \textasciitilde29°C & Moderate heat reference \\
P99 (Extreme heat) & 30.1°C & Extreme heat threshold \\
\end{longtable}

\subsubsection{Table S2: Lag Sensitivity Analysis (Intermediate
Level)}\label{table-s2-lag-sensitivity-analysis-intermediate-level}

\begin{longtable}[]{@{}llll@{}}
\toprule\noalign{}
Max Lag & Heat RR (P99) & Cold RR (P1) & Notes \\
\midrule\noalign{}
\endhead
\bottomrule\noalign{}
\endlastfoot
7 days & 1.181 & 1.080 & Heat peaks, cold delayed \\
14 days & 1.162 & 1.158 & Effects accumulating \\
\textbf{21 days} & \textbf{1.153} & \textbf{1.224} & \textbf{Baseline
specification} \\
28 days & 1.147 & 1.279 & Cold continues to accumulate \\
\end{longtable}

\subsubsection{Table S3: Heterogeneity
Assessment}\label{table-s3-heterogeneity-assessment}

\begin{longtable}[]{@{}
  >{\raggedright\arraybackslash}p{(\linewidth - 10\tabcolsep) * \real{0.1321}}
  >{\raggedright\arraybackslash}p{(\linewidth - 10\tabcolsep) * \real{0.2453}}
  >{\raggedright\arraybackslash}p{(\linewidth - 10\tabcolsep) * \real{0.0755}}
  >{\raggedright\arraybackslash}p{(\linewidth - 10\tabcolsep) * \real{0.1698}}
  >{\raggedright\arraybackslash}p{(\linewidth - 10\tabcolsep) * \real{0.0755}}
  >{\raggedright\arraybackslash}p{(\linewidth - 10\tabcolsep) * \real{0.3019}}@{}}
\toprule\noalign{}
\begin{minipage}[b]{\linewidth}\raggedright
Level
\end{minipage} & \begin{minipage}[b]{\linewidth}\raggedright
Cochran's Q
\end{minipage} & \begin{minipage}[b]{\linewidth}\raggedright
df
\end{minipage} & \begin{minipage}[b]{\linewidth}\raggedright
p-value
\end{minipage} & \begin{minipage}[b]{\linewidth}\raggedright
I²
\end{minipage} & \begin{minipage}[b]{\linewidth}\raggedright
Interpretation
\end{minipage} \\
\midrule\noalign{}
\endhead
\bottomrule\noalign{}
\endlastfoot
Intermediate & 4,308.56 & 2,112 & \textless0.001 & 51.0\% & Moderate
heterogeneity \\
Immediate & 11,523.67 & 8,144 & \textless0.001 & 29.3\% & Low-moderate
heterogeneity \\
\end{longtable}

\subsubsection{Table S4: Harvesting Analysis (Intermediate
Level)}\label{table-s4-harvesting-analysis-intermediate-level}

\begin{longtable}[]{@{}lll@{}}
\toprule\noalign{}
Lag Horizon & Heat RR (P99) & Cold RR (P1) \\
\midrule\noalign{}
\endhead
\bottomrule\noalign{}
\endlastfoot
7 days & 1.328 {[}1.196-1.474{]} & 0.831 {[}0.545-1.267{]} \\
14 days & 1.235 {[}0.987-1.546{]} & 1.090 {[}0.960-1.237{]} \\
21 days & 1.501 {[}1.401-1.609{]} & 1.321 {[}1.239-1.407{]} \\
35 days & 1.450 {[}1.341-1.568{]} & 1.453 {[}1.359-1.553{]} \\
\textbf{Harvesting Ratio} & \textbf{-0.37} & \textbf{+3.68} \\
Interpretation & Effects increase & Delayed mortality \\
\end{longtable}

\subsection{A4: Supplementary Figures}\label{a4-supplementary-figures}

\subsubsection{Figure S1: Spatial Distribution of Heat
Effects}\label{figure-s1-spatial-distribution-of-heat-effects}

\begin{figure}

\centering{

\includegraphics[width=1\linewidth,height=\textheight,keepaspectratio]{../../phase5_outputs/maps/map1_heat_effects_intermediate.png}

}

\caption{\label{fig-map-heat}\textbf{Figure S1: Geographic distribution
of heat effects (P99) across Brazilian regions.} Relative risk of
mortality at extreme heat (99th percentile) compared to MMT. Warmer
colors indicate higher heat-related mortality risk.}

\end{figure}%

\subsubsection{Figure S2: Spatial Distribution of Cold
Effects}\label{figure-s2-spatial-distribution-of-cold-effects}

\begin{figure}

\centering{

\includegraphics[width=1\linewidth,height=\textheight,keepaspectratio]{../../phase5_outputs/maps/map2_cold_effects_intermediate.png}

}

\caption{\label{fig-map-cold}\textbf{Figure S2: Geographic distribution
of cold effects (P1) across Brazilian regions.} Relative risk of
mortality at extreme cold (1st percentile) compared to MMT. Darker
colors indicate higher cold-related mortality risk.}

\end{figure}%

\subsubsection{Figure S3: Minimum Mortality Temperature by
Region}\label{figure-s3-minimum-mortality-temperature-by-region}

\begin{figure}

\centering{

\includegraphics[width=1\linewidth,height=\textheight,keepaspectratio]{../../phase5_outputs/maps/map3_mmt_intermediate.png}

}

\caption{\label{fig-map-mmt}\textbf{Figure S3: Geographic distribution
of minimum mortality temperature (MMT).} The optimal temperature for
mortality varies across Brazil's climate zones, reflecting local
adaptation patterns.}

\end{figure}%

\subsubsection{Figure S4: Lag-Response
Structure}\label{figure-s4-lag-response-structure}

\begin{figure}

\centering{

\includegraphics[width=1\linewidth,height=\textheight,keepaspectratio]{../../phase5_outputs/figures/fig13_lag_response_slices.png}

}

\caption{\label{fig-lag-response}\textbf{Figure S4: Lag-response
relationship for temperature effects.} The temporal evolution of
temperature-mortality associations across lag days (0-21). Heat effects
peak at short lags while cold effects accumulate over longer periods.}

\end{figure}%

\subsubsection{Figure S5: 3D Exposure-Lag-Response
Surface}\label{figure-s5-3d-exposure-lag-response-surface}

\begin{figure}

\centering{

\includegraphics[width=1\linewidth,height=\textheight,keepaspectratio]{../../phase5_outputs/figures/fig12_3d_surface_intermediate.png}

}

\caption{\label{fig-3d-surface}\textbf{Figure S5: Three-dimensional
exposure-lag-response surface.} The joint relationship between
temperature, lag, and mortality risk, showing how effects vary across
both temperature and time dimensions.}

\end{figure}%

\subsubsection{Figure S6: Descriptive Mortality Time
Series}\label{figure-s6-descriptive-mortality-time-series}

\begin{figure}

\centering{

\includegraphics[width=1\linewidth,height=\textheight,keepaspectratio]{../../phase5_outputs/figures/figD1_mortality_timeseries.png}

}

\caption{\label{fig-mortality-ts}\textbf{Figure S6: Daily elderly
mortality time series (2010-2024).} Seasonal patterns and long-term
trends in elderly mortality across Brazil.}

\end{figure}%

\subsubsection{Figure S7: Temperature
Distribution}\label{figure-s7-temperature-distribution}

\begin{figure}

\centering{

\includegraphics[width=1\linewidth,height=\textheight,keepaspectratio]{../../phase5_outputs/figures/figD2_temperature_distribution.png}

}

\caption{\label{fig-temp-dist}\textbf{Figure S7: Distribution of daily
mean temperatures across study regions.} The temperature distribution
showing the range of exposures experienced by the study population.}

\end{figure}%

\subsubsection{Figure S8: Seasonal
Patterns}\label{figure-s8-seasonal-patterns}

\begin{figure}

\centering{

\includegraphics[width=1\linewidth,height=\textheight,keepaspectratio]{../../phase5_outputs/figures/figD3_seasonal_patterns.png}

}

\caption{\label{fig-seasonal}\textbf{Figure S8: Seasonal patterns in
mortality and temperature.} Monthly averages showing the inverse
relationship between temperature and mortality across seasons.}

\end{figure}%




\end{document}
